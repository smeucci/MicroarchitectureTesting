\chapter{Introduzione}

L'elaborato prevede l'implementazione e il testing di micro-architetture, scelte sulla base dei testi degli esami di Ingegneria del Software dell'anno 2016.

Per ciascuno dei casi scelti l'obiettivo è quello di analizzare l'azzardo intrinseco, ovvero il fault model, e di identificare un'astrazione adatta a cogliere tale fault model. Sull'astrazione vengono definiti uno o più criteri di copertura.

Infine, viene realizzata una test suite, implementata grazie all'utilizzo di JUnit, che sia in grado di coprire l'astrazione. 
Le varie micro-architetture sono poi state estese per permettere l'utilizzo di Mockito nella test suite, assumendo una dipendenza in modo capzioso per le classi implementate.