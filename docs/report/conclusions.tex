\chapter{Conclusioni}

In questo elaborato è presentata un'analisi, comprensiva di implementazione e testing, di alcune micro-architetture tratte da esami di Ingegneria del Software.
Dopo aver, per ciascuna di esse, elaborato un modello di azzardo intrinseco, si è proceduto introducendo una loro modellazione mediante diagramma delle classi.
Il modello così costruito aveva lo scopo di definire un criterio di copertura che sia sufficiente a verificare le criticità dei vari sistemi.

A partire dal criterio di copertura definito, il functionality coverage, è stata definita una test suite, contenente tutti i test necessari per la sua realizzazione. L'implementazione dei test è stata effettuata tramite il framework Java Junit.
Le micro-architetture sono state poi estese tramite il framework Mockito, al fine di introdurre nuove funzionalità al sistema simulando l'implementazione di oggetti da cui dipendevano le classi da testare.