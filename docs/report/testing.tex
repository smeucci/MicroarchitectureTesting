\section{Testing}

Una volta individuato il criterio di copertura ottimale, si procede con la definizione di una \emph{test suite}, un insieme di test che, se verificati, rispettano le specifiche imposte dal criterio scelto.

\subsection{Definizione della \emph{test suite}}

Per la creazione e la definizione della \emph{test suite} è stato utile rappresentare i possibili test in una tabella. Nella tabella è presente il numero del test (un progressivo identificativo del singolo test), il titolo (cosa riguarda il test), una descrizione (in cui viene esplicitato il risultato atteso del test) ed un indicatore per la sua corretta esecuzione o meno.

I test definiti verificano le funzionalità principali dei sistemi in esame, andando a coprire nella loro interezza le criticità esposte nel \emph{fault model}.

\begin{table}[h!]
\caption{Test suite - Termostato}
\centering % used for centering table
\begin{tabular}{|p{1cm}|p{3cm}|p{7cm}|p{1cm}|} % centered columns (4 columns)
\hline\hline %inserts double horizontal lines
\textbf{Num} & \textbf{Titolo} & \textbf{Descrizione} & \textbf{Stato} \\ [0.5ex] % inserts table
%heading
\hline % inserts single horizontal line
1 & Verifica stato iniziale & Inizialmente lo stato del soggetto osservato deve essere impostato su READY & OK \\ \hline% inserting body of the table
2 & Verifica aggiornamento della temperatura & Il termostato si comporta in modo differente in base alla temperatura rilevata dal componente, aggiornando lo stato del controllore e attuando la corretta strategia. Se la temperatura rilevata è minore a quella impostata dal termostato, lo stato del controllore è impostato a ON; se sono uguali è impostato a READY; altrimenti a OFF. & OK \\ \hline
3 & Verifica \emph{observers} iniziali & Inizialmente il numero degli \emph{observers} associati ad un soggetto è 0. & OK \\ \hline
4 & Aggiunta \emph{observer} & Quando un nuovo \emph{observer} è associato ad un soggetto, il numero degli \emph{observers} nella lista del soggetto aumenta correttamente di 1. & OK \\ \hline
5 & Rimozione \emph{observer} & Quando un \emph{observer} è rimosso da un soggetto, il numero di elementi nella sua lista diminuisce di 1. & OK \\ \hline
6 & Notifica agli \emph{observers} & Quando la temperatura rilevata dal dispositivo si aggiorna (cambia), tutti gli \emph{observers} presenti nella lista del dispositivo vengono notificati. & OK \\ [1ex] % [1ex] adds vertical space
\hline %inserts single line
\end{tabular}
\label{table:observerstrategy} 
\end{table}

\begin{table}[h!]
\caption{Test suite - Gestione libretto universitario}
\centering % used for centering table
\begin{tabular}{|p{1cm}|p{3cm}|p{7cm}|p{1cm}|} % centered columns (4 columns)
\hline\hline %inserts double horizontal lines
\textbf{Num} & \textbf{Titolo} & \textbf{Descrizione} & \textbf{Stato} \\ [0.5ex] % inserts table
%heading
\hline % inserts single horizontal line
1 & Creazione di un esame & Un esame è creato correttamente con il nome desiderato, i CFU impostati ed il voto impostato a -1 (valore di default) & OK \\ \hline% inserting body of the table
2 & Inserimento di un esame nel libretto & Un esame è correttamente inserito nel libretto. L'inserimento è effettuato mediante una copia difensiva dell'oggetto di tipo Esame. La lista degli esami contenuti nel libretto deve dunque aumentare di uno. & OK \\ \hline
3 & Associazione Studente-Libretto &Un libretto può essere associato ad uno studente. L'associazione è effettuata usando una copia difensiva dell'oggetto Libretto. & OK \\ \hline
4 & Assegnazione di un voto ammesso & Un professore può associare un voto ad un esame. Quando il professore imposta il voto a un esame, se questo si trova nel range tra 18 e 30, il valore del voto associato all'esame deve cambiare in accordo all'assegnamento. Il professore è una classe \emph{mocked}. & OK \\ \hline
5 & Assegnazione di un voto non ammesso & Quando un professore imposta un voto fuori dal \emph{range} prefissato (da 18 a 30) ad un esame, deve essere lanciata un'eccezione.  & OK \\ \hline
6 & Calcolo della media & Quando è richiesto il calcolo della media dei voti degli esami di uno studente, una copia del libretto è usata all'oggetto Corso di Laurea, effettuando dunque una copia difensiva. Tutti gli esami con un voto diverso da -1 (il valore di default) devono essere considerati nel calcolo della media. Verificare che la media calcolata automaticamente sia corretta. & OK \\ \hline
7 & Rimozione di un esame & Un esame può essere rimosso da un libretto. Da quel momento non deve più essere conteggiato nel calcolo della media dei voti. & OK \\ \hline
8 & Ripristino del libretto & Effettuando un ripristino del libretto tutti gli esami al suo interno sono correttamente rimossi. & OK \\ [1ex] % [1ex] adds vertical space
\hline %inserts single line
\end{tabular}
\label{table:defensivecopy}
\end{table}


\begin{table}[h!]
\caption{Test suite - Generatore espressioni booleane}
\centering % used for centering table
\begin{tabular}{|p{1cm}|p{3cm}|p{7cm}|p{1cm}|} % centered columns (4 columns)
\hline\hline %inserts double horizontal lines
\textbf{Num} & \textbf{Titolo} & \textbf{Descrizione} & \textbf{Stato} \\ [0.5ex] % inserts table
%heading
\hline % inserts single horizontal line
1 & Aggiunta di un componente ad una Variabile & La classe Variabile estende la classe astratta Componente. Quando il metodo \emph{add()} viene invocato su un'istanza della classe Variabile deve essere lanciata un'eccezione. & OK \\ \hline% inserting body of the table
2 & Aggiunta di un componente ad un operatore & La classe Operatore estende la classe astratta Component. Quando il metodo \emph{add()} viene invocato su un'istanza di una classe che estende la classe Operator, i componenti passati come parametri devono essere aggiunti alla lista dei componenti dei figli (senza lanciare alcuna eccezione). & OK \\ \hline
3 & Generatore di espressioni concrete & Quando un'espressione (Variable, And, Or, Not, Parenthesis) è costruita, deve avere gli attributi impostati nel modo corretto. & OK \\ \hline
4 & Valutazione e output di un'espressione & Una generica espressione booleana deve essere correttamente valutata per tutti i possibili valori delle variabili in essa contenute. L'output dell'espressione deve essere inoltre quello atteso, in accordo con la valutazione. I valori delle variabili sono inizializzati grazie ad una classe \emph{mocked} (ValuesInitialiser) ed il suo metodo \emph{initVar()}. & OK \\ [1ex] % [1ex] adds vertical space
\hline %inserts single line
\end{tabular}
\label{table:observerstrategy} 
\end{table}


\subsection{JUnit}

Per l'implementazione dei test definiti nella \emph{Test Suite} è stata utilizzata un framework Java per il testing di unità: JUnit\footnote{JUnit, versione 4 - http://junit.org/junit4/}.

Il \emph{test di unità} prevede di andare a verificare e testare le singole entità del mio sistema (classi o metodi) al fine si assicurarsi che le singole unità di sviluppo assolvano le funzioni seguendo i requisiti, facendo sì, in questo modo, che, integrandole in sistemi più complessi, continuino a rispettarli. 
L'idea alla base è dunque quella di valutare ogni singolo metodo (o funzionalità) del sistema in funzione dei valori attesi.

Il framework mette a disposizione degli sviluppatori un insieme di \emph{annotazioni Java}, volte a indicare i vari test da eseguire e quali operazioni compiere prima e dopo il test eseguito.

\begin{lstlisting}
import static org.junit.Assert.*;
public class ThermostatTest {
	...		
	@Before
	public void setup() {
		Strategy strategy = StrategyReady.getInstance();
		device = new Device();
		device.setSensor(sensor);
		controller = new Controller(strategy, device);	
		thermostat = new Thermostat();
		thermostat.addObject(controller);
		//Initialize thermostat to 20 degrees
		thermostat.setTemperature(20);
	}
\end{lstlisting}

Nell'esempio riportato sopra si nota la presenza di un metodo di inizializzazione (\emph{setup()}) che, grazie all'annotazione \emph{@Before} viene eseguito prima dell'esecuzione di ciascuna funzione test. Questo per consentire l'inizializzazione di variabili o attributi di classe utili per l'esecuzione del test stesso.

I metodi test (annotati opportunamente dall'annotazione \emph{@Test}) verranno eseguiti senza un ordine preciso, quindi non necessariamente vengono eseguiti nell'ordine definito. Un test fallisce se fallisce almeno uno dei metodi \emph{assert} contenuti in esso. I metodi assert sono metodi statici contenuti che effettuano una semplice comparazione tra il risultato atteso ed il risultato dell'esecuzione. 

\begin{lstlisting}
	/*
	 * Number: 1
	 * Title: Exam creation
	 * Description: An exam is created with the desired name, the desired 
	 * CFU and a default score of -1.
	 */
	@Test
	public void ExamCreationTest() {
		Exam exam = new Exam("Test Exam", 9);
		assertEquals("Exam name not correctly assigned.", "Test Exam", 
							exam.getName());
		assertEquals("Exam cfu not correctly assigned.", 9, exam.getCFU());
		assertEquals("Exam score not correctly initialised.", -1, 
							exam.getScore());	
	}
\end{lstlisting}

Nel caso precedente, che implementa il test numero 1 sul sistema della copia difensiva, viene verificato che un esame sia creato correttamente. Se una delle asserzioni fallisce, il test fallisce e vengono mostrati i messaggi specificati.

Un test anche potrebbe sollevare una eccezione: in questi casi è interessante l'uso del parametro \emph{expected} nell'annotazione dove diciamo che è attesa l'eccezione. Se il metodo non solleva quel tipo di eccezione il test sarà fallito.

\begin{lstlisting}
	/*
	 * Number: 1
	 * Title: Adding a component to a Variable.
	 * Description: Variable extends the abstract class Component. 
	 * When the add method of the Variable class 
	 * is called, an exception should be thrown.
	 * 
	 */
	@Test(expected=InvalidComponentAddingException.class)
	public void AddingComponentToVariableTest() throws 
								InvalidComponentAddingException {		
		ExprBuilder builder = new ConcreteExprBuilder();
		Component varx = builder.BuildVariable("X", true);
		Component vary = builder.BuildVariable("Y", false);
		varx.add(vary);
	}
\end{lstlisting}

Nel caso considerato è il test numero 1 sul sistema di generazione di espressioni booleane. Nel caso in cui il metodo \emph{add()} venga invocato su un'istanza della classe \emph{Variable} deve essere lanciata un'eccezione. Il test dunque è considerato passato se questa viene lanciata. 

Ciascun test presente nella Test Suite è stato implementato e verificato con JUnit, indicando per ciascuno il riferimento alla tabella. La colonna \emph{Stato} della tabella indica su i test sono passati o meno.

\subsection{Mockito}

In alcune circostanze alcune classi dipendono da valori calcolati in altre classi che potrebbero però non essere ancora state implementate o semplicemente non essere direttamente accessibili (ad esempio un servizio Web in fase di sviluppo). In questi casi è comunque opportuno procedere effettuando test sul proprio sistema, ma è necessario in qualche modo simulare la presenza della classi da cui esso dipende.

Per questo scopo viene utilizzato \emph{Mockito}\footnote{Mockito, \emph{Tasty mocking framework for unit tests in Java} - http://mockito.org/}, un framework di \emph{mocking} utilizzato nel test di unità.

Mockito consente di definire oggetti \emph{mocked}, ovvero finte implementazioni di interfacce o classi astratte per le quali vengono specificati manualmente gli output desiderati per determinate chiamate ai metodi. Al fine di sperimentare il framework, le architetture prevedono l'utilizzo di una classe \emph{mocked} al loro interno. 

Come nel caso di JUnit, anche Mockito opera mediante delle semplici annotazioni: tramite \emph{@Mock} è possibile indicare quale è l'oggetto che desideriamo implementare con Mockito, poi, tramite il metodo statico \emph{when()} è possibile indicare l'output desiderato dei suoi metodi.



Descrizione breve di mockito e a cosa serve

Come lo abbiamo usato nelle tre architetture: qualche esempio di codice preso dai tre esempi