\section{Criterio di copertura}

Il \emph{criterio di copertura} scelto per le architetture in esame è il \emph{functionality coverage}.

Il \emph{functionality coverage}, \emph{copertura di funzionalità}, indica che i test devono andare a verificare tutte le proprietà e le funzionalità previste nei requisiti.

Un altro criterio di copertura preso in considerazione era il \emph{function coverage} che prevede di andare a definire test in cui tutte le funzioni definite nel sistema siano state chiamate almeno una volta. In architetture così limitate però ciò si riduceva a un tipo di copertura, la \emph{statement coverage}, ancora più stringente, in cui i test devono verificare ciascuna linea di codice almeno una volta. 

Molto più interessante, nei casi presi in esame, era definire test che verificassero le già limitate funzionalità del sistema.